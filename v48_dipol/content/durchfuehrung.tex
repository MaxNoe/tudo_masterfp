\section{Messvorgang}
\label{sec:messvorgang}

Ist die Probe für eine genügend lange Zeit $t \gg τ(T)$ dem elektrischen Feld ausgesetzt,
sind bei Temperatur $T_p$ etwa $L(T_p) \approx \frac{m E}{3 k_B T}$ Dipole in Richtung des Feldes ausgerichtet.
Wird das Feld konstant gehalten und dabei die Probe abgekühlt so wird die Relaxationszeit $τ(T)$ sehr groß.
Dadurch bleibt dann auch bei abschalten des externen Feldes die Polarisation beibehalten.
Sind keine freien Elektronen auf dem Kondensator vorhanden so lässt sich beim langsamen erwärmen der Probe der sogennnate Depolarisationsstrom messen.
Die Depolarisationsstromdichte $j(T)$ lässt sich nach \eqref{eq:dichte} durch $L(T_p)$, die Relaxationsrate
$\dd{N}{t}$ und das Dipolmoment $m$ ausdrücken:
\begin{equation}
  \label{eq:strom}
  j(T) = L(T_p) m \dd{N}{T}
\end{equation}


Die Relaxationsrate lässt sich durch die einfache Differentialgleichung
\begin{equation}
  \dd{N}{t} = - \frac{N}{τ(T)}
\end{equation}
beschreiben.
Nach einsetzen in \eqref{eq:strom} und Näherung bei gleichbleibender Erwärmung ergibt sich für die Depolarisationsstromdichte
\begin{equation}
  \label{eq:sol}
  j(T) =
    \frac{p^2 E}{3 k_B T_p}
    \frac{N_p}{τ(T)}
    \exp\!\left(
      -\frac{1}{b} \int_{T_0}^{T} \frac{\dif{T'}}{τ(T')}
    \right).
\end{equation}


Nach einsetzen von Gleichung \ref{eq:time} und Näherung für kleine $T - T_0$ folgt
\begin{equation}
  \label{eq:sol_approx}
  j(T) = \frac{p^2 E N_p}{3 k_B T_p τ_0} \cdot \E^{- \frac{W}{k_B T}}.
\end{equation}

Anwendung des natürlichen Logarithmus auf \eqref{eq:sol_approx} ergibt einen linearen Zusammenhang zwischen $\ln{j(T)}$ und $\sfrac{1}{T}$:
\begin{equation}
  \label{eq:linear}
  \ln{j(T)} = \underbrace{\ln\!\left(
      \frac{p^2 E N_p}{3 k_B T_p τ_0}
    \right)}_{\mathclap{\text{konst}}}
    \overbrace{- \frac{W}{k_B}}^{\mathclap{\text{Steigung}}}
    \cdot \frac{1}{T}
\end{equation}

Die Aktivierungsenergie ergibt sich damit aus der Steigung der Geraden $a$ zu
\begin{equation}
  W = - a \cdot k_B
\end{equation}

In Abbildung \ref{fig:lnj} ist der Logarithmus der Stromdichte gegen $\sfrac{1}{T}$ aufgetragen.

\begin{figure}
  \includegraphics{fit_linear.pdf}
  \caption{%
    Logarithmus des Stroms gegen den Kehrwert der Temperatur für die gemessenen Werte.
    Die für die Berechnung der Ausgleichsgeraden genutzten Werte sind in violett markiert.
  }
  \label{fig:lnj}
\end{figure}

Die markierten Punkte wurden für die Bestimmung der linearen Ausgleichsgerade der Form
\begin{equation}
  f(x) = a (x - x_0) + b
\end{equation}
genutzt.

Für $x_0$ wird der Mittelwert von $\sfrac{1}{T}$ im gewählten Bereich gewählt, dies minimiert die Korrelation zwischen $a$ und $b$.
Für die Parameter ergibt sich:
\begin{align}
  \input{fit_parameters.tex}
\end{align}
Damit ergibt sich für die Aktivierungsenergie:
\begin{equation}
  \input{activation_work.tex}
\end{equation}

Die zweite Möglichkeit zur Messung der Aktivierungsenergie $W$ benutzt den gesamten Kurvenverlauf des gemessen Stroms.
Für die Gesamtpolarisation $P$ gilt die Differentialgleichung
\begin{equation}
  \dd{P}{t} = - \frac{P(t)}{\tau{T(t)}}.
  \label{eq:pol}
\end{equation}
Der Strom, welcher durch die Polarisationänderung hervorgerufen wird, hängt vom Probenquerschnitt $F$ über
\begin{align}
  i(t) &= F \dd{P}{t} \\
  \shortintertext{mit}
  \int_t^{\infty} i(t) \d{t} &= -FP(t)
  \label{eq:i}
\end{align}
ab.
Durch Integration von  \eqref{eq:i} und einsetzen in \eqref{eq:pol} folgt:
\begin{equation}
  \tau(T(t)) = -\frac{F P(t)}{i(t)}.
  \label{eq:tau_t}
\end{equation}
Nimmt man einen linearen zusammenhang zwischen der Zeit $t$ und der Temperatur $T$ an, so kann man die Integrationsvariable in \eqref{eq:i} durch $T$ ersetzen.
Zusammen mit Gleichung \eqref{eq:time} ergibt sich \eqref{eq:final}. Da sich das Integral ab einer Temperatur mit $i(T') \approx 0$ nicht mehr ändert,
kann die obere Integrationsgrenze durch $T'$ ersetzt werden.
\begin{equation}
  W = k_B T \ln \frac{\int_T^{T'} i(T)}{ i(T) \tau_0}
  \label{eq:final}
\end{equation}

\begin{figure}
  \centering
  \includegraphics{fit_non_linear.pdf}
  \caption{Exp-Fit an die gemessenen Werte außerhalb des Maximums}
  \label{fig:fit_non_linear}
\end{figure}

\begin{figure}
  \centering
  \includegraphics{fit_non_linear_corr.pdf}
  \caption{Fit-Ergebnis von den Messwerten abgezogen.}
  \label{fig:fit_non_linear}
\end{figure}
