\section{Diskussion}\label{sec:Diskussion}


Ein Wert aus \cite{dipoles} für die Aktivierungsenergie beträgt:
\begin{equation}
  W_\text{lit} = \SI{0.66}{\electronvolt} \quad \text{\cite{dipoles}}
\end{equation}
Dies ist in der selben Größenordnung wie der von uns gemessene Wert von
\begin{equation}
  \input{activation_work_1.tex}
\end{equation}
bzw.
\begin{equation}
  \input{activation_work_2.tex} .
\end{equation}

Die relative Abweichung zum Wert aus \cite{dipoles} beträgt demnach $\sfrac{\Delta W}{W}  = \SI{24.2}{\percent}$.
Allerdings lassen sich die Werte nur bedingt vergleichen, da die Aktivierungsenergie von der Dotierung des Materials abhängt.



Die von uns bestimmte Relaxationszeit von
\begin{equation}
  \input{tau.tex}
\end{equation}
 weicht jedoch deutlich und um einige Größenordnungen vom Literaturwert ab.
\begin{equation}
  \tau_\text{lit} = \SI{4E-10}{\second} \quad \text{\cite{dipoles}}
\end{equation}
Eine große Fehlerquelle bei diesem Versuch wird die Messung des Stromes durch das
Ampermeter sein.
Die Anzeige schien sich während der Messung instabil zu verhalten.
Auch ist die Wahl der Punkte, die in den Fit eingehen und die Wahl von $T'$ eher beliebig wirkt sich aber stark auf das Endergebniss aus.
