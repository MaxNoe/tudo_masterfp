\section{Versuchsaufbau}

Der Versuchsaufbau besteht im wesentlichen aus der Strahlungsquelle, einer Goldfolie und einem Surface-Barrier Detektor.
Der Aufbau befinded sich in einer Vakuumkammer welche durch eine Drehschieberpumpe evakuiert werden kann.
Die Alpha-Strahlen aus der Quelle werden durch \SI{2}{\mm} Schlitzblenden kollimiert bevor sie auf die Goldfolie treffen.


\section{Versuchsdurchführung}

Als Quelle für Alpha-Teilchen wird in diesem Versuch \ce{Am241} genutzt. Als Vorbereitungsaufgabe sollte das Bremsvermögen
der Alpha-Teilchen in Luft berechnet bestimmt werden. Daraus solte dann der Kammerdruck bestimmt werden ab dem  ein Energieverlust
bemerkbar wird. In Abbildung~\ref{fig:bethe} ist das Bremsvermögen dargestellt.

\begin{figure}
  \centering
  \includegraphics[width=0.9\textwidth]{./build/plots/bethe_air.pdf}
  \caption{Das Bremsvermögen der Alpha-Teilchen gegen den Kammerdruck. Die Atmosphäre wurde hier als reiner Stickstoff mit einer Ionisationsenergie von \SI{5.408}{\MeV} genähert.}
  \label{fig:bethe}
\end{figure}

Im vorliegenden Versuchaufbau befinden sich etwa \SI{10}{\cm} Atmosphäre zwischen Quelle und Detektor. Ein Großteil der Alpha-Teilchen würde also schon bei einem Kammerdruck
von ungefähr \SI{100}{\milli \bar} nicht mehr beim Detektor ankommen.


An den Surface-Barrier Detektor wird ein Osziloskop angeschlossen. Nach Vorverstärkung werden Pulse sichtbar. Exemplarisch ist ein Puls in Abbidlung~\ref{fig:pulse} dargestellt.


\begin{figure}
  \centering
  \includegraphics[width=0.9\textwidth]{./build/plots/pulse_with_amp.pdf}
  \caption{Ein vom Oszilloskop aufgenommener Puls.}
  \label{fig:pulse}
\end{figure}
