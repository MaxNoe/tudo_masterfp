\section{Versuchsaufbau}

Der Versuchsaufbau besteht im wesentlichen aus der Strahlungsquelle, einer Goldfolie und einem Surface-Barrier Detektor.
Der Aufbau befindet sich in einer Vakuumkammer welche durch eine Drehschieberpumpe evakuiert werden kann.
Die α-Teilchen aus der Quelle werden durch \SI{2}{\mm} Schlitzblenden kollimiert bevor sie auf die Goldfolie treffen.
Der Versuchsaufbau ist in \autoref{fig:aufbau} schematisch dargestellt.

\begin{figure}
  \centering
  \includegraphics[scale=1.2]{aufbau.pdf}
  \caption{%
    Schematische Darstellung des Versuchsaufbaus~\cite{anleitung_v16}, Maße in Millimetern.
  }\label{fig:aufbau}
\end{figure}

\section{Versuchsdurchführung}
Als Quelle für Alpha-Teilchen wird in diesem Versuch \ce{^{241}Am} genutzt.
Dieses Isotop besitzt drei Alpha-Linien\cite{alpha-spectrum}:
\begin{center}
  \begin{tabular}{S[table-format=1.3] @{\,} s S[table-format=1.2]}
    \toprule
    \multicolumn{2}{c}{Energie} & {Anteil} \\
    \midrule
    5.486 & MeV & 0.85 \\
    5.443 & MeV & 0.13 \\
    5.388 & MeV & 0.01 \\
    \bottomrule
  \end{tabular}
\end{center}

Die Probe hatte im Oktober 1994 eine Aktivität von $A_0 = \SI{330}{\kilo\becquerel}$,
hieraus ergibt sich mit der Halbwertszeit von \ce{^{241}Am} von \num{432.5} Jahren eine Aktivität zur Zeit der Versuchsdurchführung \SI{318}{\kilo\becquerel}.
In den folgenden Berechnungen wird nur die Hauptlinie des Spektrums bei \SI{5.486}{MeV} berücksichtigt.

\begin{equation}
  A = A_0 \cdot \E^{
    - \ln(2) \frac{t}{T_{\sfrac{1}{2}}}
  } = \SI{318}{\kilo\becquerel}.
\end{equation}

Als Vorbereitungsaufgabe sollte das Bremsvermögen
der Alpha-Teilchen in Luft berechnet bestimmt werden. Daraus solte dann der Kammerdruck bestimmt werden ab dem  ein Energieverlust
bemerkbar wird. In Abbildung~\ref{fig:bethe} ist das Bremsvermögen dargestellt.

\begin{figure}
  \centering
  \includegraphics[width=0.9\textwidth]{./build/plots/bethe_air.pdf}
  \caption{Das Bremsvermögen der Alpha-Teilchen gegen den Kammerdruck. Die Atmosphäre wurde hier als reiner Stickstoff mit einer Ionisationsenergie von \SI{5.408}{\MeV} genähert.}
  \label{fig:bethe}
\end{figure}

Integriert man den Kehrwert des Energieverlusts von der Teilchenenergie $E_{\alpha}$ bis $0$ erhält man die Reichweite der Alphastrahlung.
In \autoref{fig:alpha_range} ist die Reichweite der Alpha-Strahlung für mehrere Drücke aufgetragen.

\begin{figure}
  \centering
  \includegraphics{range_alpha.pdf}
  \caption{Reichweite von Alpha-Strahlung für verschiedene Drücke in einer Stickstoff-Atmosphäre.}
  \label{fig:alpha_range}
\end{figure}

Im vorliegenden Versuchaufbau befinden sich etwa \SI{10}{\cm} Atmosphäre zwischen Quelle und Detektor.
Eine Reichweite von \SI{10}{\centi\meter} wird bei ca.\  \SI{350}{\milli\bar} erreicht.

An den Surface-Barrier Detektor wird ein Osziloskop angeschlossen.
In \autoref{fig:pulse} sind beispielhaft zwei Pulse gezeigt.
Der obere Puls ist ohne Vorverstärker aufgenommen, der untere mit.

\begin{figure}
  \centering
  \includegraphics[width=0.9\textwidth]{./build/plots/pulses.pdf}
  \caption{%
    Vom Oszilloskop aufgenommene Pulse ohne (oben) und mit (unten) Vorverstärkung.
  }%
  \label{fig:pulse}
\end{figure}


\subsection{Bestimmung der Foliendicke}

Zunächst soll die Dicke der Goldfolie durch eine Energieverlustmessung bestimmt werden.

Hierzu wird die mittlere Amplitude der Pulse des
Surface-Barrier-Detektors für verschiedene Drücke bestimmt.
Die Messung wird einmal mit und einmal ohne Folie durchgeführt.

Da beim Surface-Barrier-Detektor die Höhe der Pulse proportional zur Energie der eintreffenden Teilchen ist,
kann über den Unterschied zwischen der Messung mit und ohne Folie den Energieverlust $\increment E$ ermitteln.

In \autoref{fig:thickness} sind die beiden Messreihen aufgetragen.
Über eine lineare Extrapolation wird die Energie der α-Teilchen bei einem Druck von $0$ bestimmt, also für keinerlei Energieverlust in der Atmosphäre.

Für die Parameter der linearen Regression ergibt sich:
\input{fit_results.tex}
Der Energieverlust ergibt sich zu:
\begin{equation}
  \increment E = E_α \cdot \left(1 - \frac{b_\text{ohne}}{b_\text{mit}}\right)
  = \input{delta_E.tex}
\end{equation}

Mithilfe 
\begin{equation}
  d_\text{Gold} = \input{thickness.tex} .
\end{equation}

\begin{figure}
  \centering
  \includegraphics{gold_thickness.pdf}
  \caption{%
    Mittlere Pulshöhe am Oszilloskop gegen Kammerdruck für die Messungen mit bzw.\ ohne Folie.
    Die durchgezogenen Linien zeigen das Ergebnis der linearen Regression.%
  }\label{fig:thickness}
\end{figure}
