\section{Diskussion}

Bei der Versuchsdurchführung sind einige mögliche Problemquellen aufgefallen.
Zunächst musste eine Spannungsstabilisierendes Einheit and die Spannungsversorgung angeschlossen werde.
Die Netzversorgung im Labor zu schien zu instabil. Ohne Stabiliserung konnten keine konsistenten Zählraten gemessen werden.
Die Pulse aus dem Surface-Barrier Detektor können durch einen Verstärker geschleift werden, was sich jedoch für eine reine Zählratenmessung
als unnötig herausstellte.
Für die Berechnung der Pulshöhe wird der Verstärker benutzt um ein klareres Maximum der Kurven zu sehen.
Bei der Abschätzung der Foliendicke wird angenommen, dass die Verstärkung der Pulsamplituden linear erfolgt.
Dies wurde jedoch nicht explizit von uns getestet und könnte eine Fehlerquelle bei der Bestimmung darstellen.
Unser gemessener Wert von $d_\text{Gold} = \input{thickness.tex}$ weicht signifikant um etwa \SI{1.2}{\micro\meter} vom wahren Wert \SI{1.2}{\micro\meter} ab.
Mit der wahren Dicke wird auch in den weiteren Aufgaben gerechnet.

Die Messung des Streuquerschnitts scheint fehlgeschlagen zu sein. Die Abweichung der gemessenen Werte von der Theoriekurve in \autoref{fig:cross_section} ist mehr als deutlich.
Aufgrund der Abweichung wurde eine Leermessung ohne Goldfolie durchgeführt, die zu ähnlichen Werten und Winkelabhängigkeiten führte.
Wir vermuten deshalb, dass  die Messung von Untergrund dominiert wird.
Wir schlagen deshalb vor eine Leermessung ins Versuchprogramm aufzunehmen, um Untergrund abziehen zu können.
Desweiteren kann so die Teilchenrate am Detektor ohne Streuung direkt gemessen werden und muss nicht über \eqref{eq:rate} abgeschätzt werden.
