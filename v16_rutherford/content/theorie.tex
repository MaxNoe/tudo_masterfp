\section{Einleitung}
\label{sec:einleitung}
In diesem Versuch sollen der Streuprozess von Alpha-Teilchen an Goldfolie untersucht werden. Im folgenden werden zunächst einige theoretische
Grundlagen beschrieben und anschließend der Versuchsaufbau und die Durchführung erklärt. Im Anschluss werden die Ergebnisse dargestellt.



\section{Theorie der Rutherfordstreuung}
\label{sec:theorie}

Rutherfordstreuung entsteht wenn schwere, geladene Teilchen mit dem Kernpotential eines Atoms wechselwirken.
Der Name kommt von dem bekannten Experiment welches Ernest Rutherford (1871–1937)  um
1910 mit Alpha-Strahlung und Goldfolie durchführte. Rutherford schoss Alpha-Teilchen auf eine Goldfolie
und nahm die Raumrichtungen der gestreuten Teilchen auf. Ein großer Teil der Teilchen durchdringt
die Folie ohne merkbare Richtungsänderung oder Energieverlust. Ein kleinerer Teil der Strahlung wird
in verschiedene Richtungen abgelenkt.
Rutherford schloss daraus, dass die Atome einen kleinen aber schweren Kern enthalten,
welcher mit einem Teil der Alpha-Strahlen wechselwirkt.
Die Ergebnisse aus Rutherfords Experiment führten zur Ablösung des bis dahin verbreiteten Thomson'schen Atommodells.
Beim Durchgang von geladenen Teilchen durch ein Material wechselwirken die Teilchen sowohl mit den
Elektronen der Atomhülle als auch mit dem Kernpotential selbst.

\subsection{Bethe-Bloch-Formel}
\label{sub:bethe}

Da Alpha-Teilchen wesentlich schwerer als Elektronen sind,
ist der Energieverlust und die Richtungsänderung eher klein.
Die Bethe-Bloch-Formel beschreibt den Energieverlust von geladenen Teilchen beim Durchgang durch ein
Material wenn sie mit Elektronen wechselwirken.
Im nicht relativistischem Fall sieht sie aus wie in \eqref{eq:bethe} dargestellt.
Dabei ist $x$ die im Material zurückgelegte Wegstrecke, $m_0$ die Ruheenergie des Elektrons,
$v$ die Geschwindigkeit des einfallenden Teilchens, $I$ die Ionisationsenergie der Alpha-Strahlung,
$n$ die mittlere Elektronendichten pro Volumeneinheit und$z$ die Kernladungszahl des Teilchens.

\begin{equation}
  \label{eq:bethe}
  \dd{E}{x} = \frac{4 \pi^4 z^2 n e^4}{m_0 v^2 (4 \pi \epsilon_0)^2} \ln \frac{2 m_0 v^2}{I}
\end{equation}

Die Formel gilt nur in Energiebereichen in denen durch Kollisionen keine Sekundärteilchen entstehen können.
Außerdem wird die Bewegung der Elektronen im Probenmaterial vernachlässigt.


\subsection{Streuquerschnitt}
Die rutherfordsche Streuquerschnitt beschreibt die Streuung zweier spinloser punktförmiger Teilchen. Da in diesem Experiment nur Geschwindigkeiten
$v \ll c_0$ auftreten, reicht es den nicht-relativistischen Fall zu betrachten.

Nach Fermis Goldener Regel ist der differentielle Wirkungsquerschnitt $\dd{\sigma}{\Omega}$ proportional zum Quadrat der Übergangsamplitude $M$ multipliziert mit dem Phasenraumfaktor $\rho$.

\begin{equation}
  \dd{\sigma}{\Omega} \propto \abs{M}^2 \rho
\end{equation}

Die Übergangsamplitude ergibt sich hier aus der Coulomb-Wechselwirkung und ist abhängig von der Ladung der Streupartner.


\begin{equation}
  M \propto Z_1 Z_2
\end{equation}


Aus geometrischen Überlegungen zur Raumwinkelgröße und zum Phasenraumfaktor folgt die Abhängikeit zum Streuwinkel $\theta$ im Vorfaktor $\kappa$

\begin{equation}
  \kappa \propto \frac{1}{\sin^4{\frac{\theta}{2}}}
\end{equation}

Insgesamt folgt die Rutherfordsche Streuformel wie in der Anleitung zu

\begin{equation}
  \label{ruther}
  \dd{\sigma}{\Omega} (\theta) = \frac{1}{(4 \pi \epsilon_0)^2} \left( \frac{Z_1 Z_2}{4 E_{\alpha}} \right)^2 \frac{1}{\sin^4{\frac{\theta}{2}}}
\end{equation}

wobei $E_{\alpha}$ die mittlere kinetische Energie der einfallenden Alpha Strahlung ist.
