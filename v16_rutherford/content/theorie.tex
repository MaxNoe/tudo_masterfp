\section{Theorie der Rutherfordstreuung}
\label{sec:theorie}

Rutherfordstreuung entsteht wenn schwere geladene Teilchen mit dem Kernpotential eines Atoms wechselwirken.
Der Name kommt von dem bekannten Experiment welches Ernest Rutherford (1871 - 1937)  um
1910 mit Alpha-Strahlen und Goldfolie durchführte. Rutherford schoss Alpha-Teilchen auf eine Goldfolie
und nahm die Raumrichtungen der gestreuten Teilchen auf. Ein großer Teil der Teilchen durchdringt
die Folie ohne merkbare Richtungsänderung oder Energieverlust. Ein kleinerer Teil der Strahlung wird
in verschiedene Richtungen abgelenkt. Rutherford schloss daraus, dass die Atome einen, kleinen
aber schweren, Kern enthalten welcher mit einem Teil der Alpha-Strahlen wechselwirkt.
Die Ergebnisse aus Rutherfords Experiment führten zur Ablösung des bis dahin verbreiteten
Thomsonschen Atomodells.
Beim durchgang von geladenen Teilchen durch ein Material, wechselwirken die Teilchen sowohl mit den
Elektronen der Atomhülle als auch mit dem Kernpotential selber.

\subsection{Bethe-Bloch-Formel}
\label{sub:bethe}

Da Alpha-Teilchen wesentlich schwerer als Elektronen sind,
ist der Energieverlust und die Richtungsänderung eher klein.
Die Bethe-Bloch-Formel beschreibt den Energieverlust von geladene Teilchen beim Durchgang durch ein
Material wenn sie mit Elektronen wechselwirken.
Im nicht relativistischem Fall sieht sie aus wie in \eqref{eq:bethe} dargestellt.
Dabei ist $x$ die im Material zurückgelegte Wegstrecke, $m_0$ die Ruheenergie des Elektrons,
$v$ die Geschwindigkeit des einfallenden Teilchens, $I$ die Ionisationsenergie der Alpha-Strahlung,
$n$ die mittlere Elektronendichten pro Volumeneinheit und$z$ die Kernladungszahl des Teilchens.

\begin{equation}
  \label{eq:bethe}
  \dd{E}{x} = \frac{4 \pi^4 z^2 n e^4}{m_0 v^2 (4 \pi \epsilon_0)^2} \ln \frac{2 m_0 v^2}{I}
\end{equation}

Die Formel gilt nur in Energiebereichen in denen durch Kollisionen keine Sekundärteilchen entstehen können.
Außerdem wird die Bewegung der Elektronen im Probenmaterial vernachlässigt.


\subsection{Streuquerschnitt}
Nach Fermis Goldener Regel ist der differentielle Wirkungsquerschnitt $\dd{\sigma}{\Omega}$ proportional zum Quadrat der Übergangsamplitude $M$ multipliziert mit dem Phasenraumfaktor $\rho$.

\begin{equation}
  \dd{\sigma}{\Omega} \propto \abs{M}^2 \rho
\end{equation}

Der Vorgang ist isotropisch in alle Raumrichtungen. Der phasenraumfaktor
