\section{Diskussion}\label{sec:Diskussion}

Die frequenzabhängige Messung von $L$, $C$ und $R$ in Abbildung~\ref{fig:lcrg} entspricht in den niederen Frequenzen ganz der Physikalischen Erwartung.
widerstand, Kapazität und Induktivität machen sich bei diesen Frequenzen noch nicht groß bemerkbar. Die Messung der Induktivität zeigt große Schwankungen
welche wir auf Störungen im Kabel oder der Umgebung zurückführen. Leider ließ sich der interessante Messbereich von \SIrange[range-phrase={~bis~}]{e4}{e5}{\hertz}
mit dem benutzten Frequenzgenerator nicht ansteuern.

Die Messung der Dämpfungskonstante zeigt das erwartete Verhalten. Das lange Kabel dämpft hohe Frequenzen deutlich stärker als das kurze.

Die berechneten Kabellängen aus Tabelle~\ref{tab:length} geben die echte Kabellänge gut, aber nicht exakt, wieder. Abweichung entstehen durch die Verschmierung
des NIM-Pulses im Kabel oder durch eine falsche Annahme der relative Permittivität von $\epsilon_r = \num{2.25}$. Die Abweichung ist bei dem \SI{20}{\meter} Kabel
am größten. Entweder das Kabel ist nicht so lang wie versprochen oder die Permittivität ist deutlich von der Annahme verschieden.

Die Werte der Abschlussmessung erscheinen uns als sinnvoll, lassen sich aber nicht überprüfen da die Bauteile in den Abschlüssen uns nicht bekannt sind.
Zumindest scheinen die Größenordnungen und Fehler im Rahmen von physikalisch sinnvollen Bereichen zu liegen.

Mehrfachreflektion an der Störstelle war sehr deutlich auf dem osziloskop zu sehen. Die Positionen und Formen der Signale entspricht dem theoretischen Impulsfahrplan
wie in Abbildung~\ref{fig:multiple_reflection} zu erkennen.

 
