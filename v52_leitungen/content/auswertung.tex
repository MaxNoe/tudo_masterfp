\section{Auswertung}
\label{sec:Auswertung}

\subsection{Messung der Kabeleigenschaften}

\begin{figure}
  \centering
  \includegraphics{lcrg.pdf}
  \caption{Widerstand $R$, Kapazität $C$, Induktivität $L$ und Querleitwert $G$ für das \texttt{RG-58}-Kabel mit einer Länge von \SI{85}{\meter} in Abhängigkeit der Frequenz der angelegten Sinusspannung.}
  \label{fig:lcrg}
\end{figure}


\subsection{Messung der Dämpfungskonstante}

Für die Messung der frequenzabhängingen Dämpfung wird eine Rechteckspannung
jeweils über ein kurzes Kabel (ca.\ \SI{25}{\centi\meter}) und das zu untersuchende \SI{85}{\meter}-Kabel in ein Oszilloskop gespeist.

Da eine Rechteckspannung durch eine Fourierreihe aus vielen Frequenzen erzeugt wird,
kann die Dämpfungskonstante in einer Messung für viele Frequenzen bestimmt werden.
Hierzu werden die gemessenen Spannungen im Oszilloskop Fourier-transformiert.

Die eingespeisten Rechteckspannungen sind in \autoref{fig:attenuation_signal} dargestellt.
Es lässt sich erkennen, dass die Form für das kurze Kabel deutlich \enquote{eckiger} ist als bei dem Signal, dass durch das \SI{85}{\meter}-Kabel geleitet wurde.
Dies deutet auf eine Abschwächung der höheren Frequenzen hin.

\begin{figure}
  \centering
  \includegraphics{attenuation_signal.pdf}
  \caption{%
    Rechteckspannung im Oszilloskop nach Leitung durch das kurze bzw.\ \SI{85}{\meter}-Kabel.%
  }\label{fig:attenuation_signal}
\end{figure}

In \autoref{fig:attenuation_fft} sind die Fourier-transformierten Spannungswerte gegen die Frequenzen aufgetragen.
Die Dämpfungskonstante wird für jedes lokale Maxima im Frequenzraum bestimmt,
da das Oszilloskop bereits die Amplitude $A$ in \si{\deci\bel} ermittelt, ergibt sich die Dämpfungskonstante $α$ zu
\begin{equation}
  α / \si{\deci\bel} = (A_{\SI{85}{\meter}} - A_\text{kurz}) / \si{\deci\bel}
\end{equation}

\begin{figure}
  \centering
  \includegraphics{attenuation_fft.pdf}
  \caption{%
    Fourier-transformiertes Signal mit lokalen Maxima für beide Kabel, sowie die für die lokalen Maxima bestimmte Dämpfungskonstante $α$.
  }\label{fig:attenuation_fft}
\end{figure}

\subsection{Bestimmung der Kabellänge}

\begin{figure}
  \centering
  \includegraphics{length_measurement.pdf}
  \caption{%
    Signal des NIM-Pulses mit Reflektion, der Zeitabstand zwischen den beiden Pulsen wird zur Bestimmung der Kabellänge genutzt.
  }\label{fig:length}
\end{figure}

\begin{table}
  \centering
  \caption{%
    Laufzeitunterschied $\increment t$ und
    Kabellänge $l$ für die drei untersuchten Kabel
  }\label{tab:length}
  \input{build/length.tex}
\end{table}

\subsection{Vermessung von unbekannten Abschlusswiderständen}
Es sollte der Spannungsverlauf für verschiedene Abschlüsse gemessen werden. Die Abschlüsse können charachteriesiert werden indem die rücklaufenden
Signale eines Rechteckpulses vermessen werden. Nach Formel 7 der Anleitung~\cite{v52}
Mithilfe des Oszilloskops wurde der Zeitverlauf der Spannung für drei unbekannte Abschlusswiderstände aufgenommen.
Die Abschlüsse befinden sich an Kästen mit durchnummerierten Anschlüssen.
Getestet wurden die Anschlüsse mit den Nummern 4 und 6 an Kasten 1. An Kasten 2 wurde Anschluss 2 gemessen.
Die aufgezeichneten Spannungsverläufe werden mit denen verglichen, welche in Tabelle 2 der Anleitung~\cite{v52} dargestellt sind.
Die Spannungskurve für Anschluss 4 und 6 von Kasten 1 ähnelt dem Verlauf von Reihe 4 in Tabelle 2.
Es handelt sich also mutmasslich um einen Abschlusses welcher aus einer Reihenschaltung aus ohmschen Widerstand und Induktivität besteht (siehe
Abbildung~\ref{fig:end_1}).
Der Anschluss 2 and Kasten 2 entspricht am ehesten dem Abschluss aus Abbildung~\ref{fig:end_2}.

\begin{figure}
    \centering
    \begin{subfigure}[b]{0.45\textwidth}
        \begin{circuitikz}[american voltages]
\draw
  % stator circuit
  (0,0) to [open, v^>=$U$] (0,3) %
  to [short, *- ] (2,3) %
  % to [R, l=$R_L$] (3,3) %
  to [L, l=$L$] (6,3) %
  to [C, l=$C$] (6,0) %

  (6,3)    to [short, -*] (10,3) %

  (0,0) to [short, *-*] (10,0);
\end{circuitikz}

        \caption{Widerstand in Reihe mit Induktivität}
        \label{fig:end_1}
    \end{subfigure}
    ~ %add desired spacing between images, e. g. ~, \quad, \qquad, \hfill etc.
    \begin{subfigure}[b]{0.45\textwidth}
        \begin{circuitikz}
\draw
  (0,0) to [C] (0,1.5) %
  to [R] (0,3); %
\end{circuitikz}

        \caption{Widerstand in Reihe mit Kapazität}
        \label{fig:end_2}
    \end{subfigure}
    \caption{Die mutmasslichen Abschlüsse der Koaxialkabel and den unmarkierten Kästen.}\label{fig:ends}
\end{figure}


\begin{figure}
  \centering
  \includegraphics{unknown.pdf}
  \caption{%
    Signalspannung am Oszilloskop für drei unbekannte Abschlusswiderstände.
    Kasten 1 der Anschluss mit der Nummer 4 in der obersten Spannungskurve. Kasten 2 jeweils Anschlüsse 2 und 6 in den
    unteren Spannungskurven.
  }\label{fig:unknown}
\end{figure}


Um die Eigenschaften der in den Abschlüssen verbauten Kapazitäten und Induktivitäten zu bestimmen,
wird der Zeitverlauf des rücklaufenden Signals vermessen wie in~\ref{ref:telegraph} beschrieben. Zunächst wird der Reflektionsfaktor $\Gamma$ bestimmt.
Dazu wird das Verhältniss aus der Spannung des hin- und rücklaufenden Signals berechnet. Die Spannungen sind als horizontale Linien in
Abbildung~\ref{fig:fit} eingezeichnet. Aus dem Reflektionsfaktor lässt sich mit dem Wellenwiderstand des Kabels $Z_0$ über
\begin{equation}
  R = - Z_0 \frac{ 1 + \Gamma}{ 1 - \Gamma}
\end{equation}
der ohmsche Widerstand des Abschlusses bestimmen.
Als hinlaufende Spannung wurde jeweils der Mittelwert der Spannungen des Plateaus des Eingangssignal gewählt. Dadurch ergibt sich
für den Widerstand von Anschluss 2 von Kasten 2 ein Wert von $\input{build/k2a2_r.tex}$. Entsprechend für Anschluss 6 an Kasten 6 ein Widerstand
von $\input{build/k1a6_r.tex}$.
Es wurde eine Exponentialfunktion an das Signal gefittet. Die angepasste Funktion ist in blauer Farbe eingezeichnet.
Die so erhaltene Zeitkonstante wurde mit Tabelle 2 aus der Anleitung in die gewünschten Größen umgerechnet.
Für die vermutete Kapazität hinter Anschluss 2 an Kasten 2 ergibt sich so $\input{build/k2a2_c.tex}$.
Die Induktivität hinter Anschluss 6 an Kasten 1 hat einen Wert von $\input{build/k1a6_l.tex}$.

\begin{figure}
  \centering
  \includegraphics{time_constant_fit.pdf}
  \caption{%
    Signalspannung am Oszilloskop für zwei der drei unbekannten Abschlusswiderständen. Eine Exponentialfunktion wurde an das reflektierte Signal
    gefittet. Das Ergebnis ist in blau eingezeichnet. Die horizontalen Linien sind die Spannungen welche für die Berechnung des Reflektionsfaktor
    benutzt wurden.
  }\label{fig:fit}
\end{figure}
\subsection{Mehrfachreflektion}

Für die Untersuchung des Verhaltens von Störstellen in Leitungen wird eine wohldefinierte
Störstelle durch die Reihenschaltung zweier Koaxialkabel mit unterschiedlicher Impededanz
erzeugt.
In diesem Versuch werden zwei \SI{10}{\meter} lange Kabel mit \SI{50}{\ohm} bzw. \SI{75}{\ohm} verwendet.
Es wird erwartet, dass es Mehrfachreflektionen des Signals, sowohl an den eigentlichen Enden
des Kabels, als auch an der \enquote{Störstelle}, dem Übergang zwischen den Kabeln gibt.

Die Situation ist im sogenannten Impulsfahrplan in \autoref{fig:impuls} dargestellt.

\begin{figure}
  \captionsetup{format=plain}
  \begin{captionbeside}{%
      Impulsfahrplan für ein \SI{20}{\meter} langes Kabel mit einer Störstelle in
      der Mitte des Kabels und geschlossenem Ende.
      Das Signal wird am  Anfang des Kabels bei $l = 0$ gemessen, nach einer Zeiteinheit
      erreicht die erste Reflektion an der Störstelle wieder den Anfang des Kabels.
      Nach vier Zeiteinheiten treffen die erste Reflektion am Ende des Kabels und die
      zweite Reflektion an der Störstelle ein. Nach 6 Zeiteinheiten treffen bereits vier
      Signale gleichzeitig am Oszilloskop ein.
    }%
    \includegraphics{impuls.pdf}
  \end{captionbeside}\label{fig:impuls}
\end{figure}

Gemessen wurde jeweils das Signal eines Rechteckspulses und eines NIM-Pulses.
Der gemessene Signalverlauf ist zusammen mit dem Impulsfahrplan in \autoref{fig:multiple_reflection} aufgetragen.
Impulsfahrplan und gemessenes Signal stimmen überein.

\begin{figure}
  \centering
  \includegraphics{multiple_reflection.pdf}
  \caption{%
    Signale einer Recheckspannung (oben) und eines NIM-Pulsers (Mitte) für eine Reihenschaltung eines \SI{50}{\ohm}-Kabels mit einem \SI{75}{\ohm}-Kabel.
    Der Impulsfahrplan (unten) stimmt mit den beobachteten Signalverläufen überein.
  }\label{fig:multiple_reflection}
\end{figure}
