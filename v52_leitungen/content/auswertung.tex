\section{Auswertung}
\label{sec:Auswertung}

\subsection{Messung der Kabeleigenschaften}

\begin{figure}
  \centering
  \includegraphics{lcrg.pdf}
  \caption{Widerstand $R$, Kapazität $C$, Induktivität $L$ und Querleitwert $G$ für das \texttt{RG-58}-Kabel mit einer Länge von \SI{85}{\meter} in Abhängigkeit der Frequenz der angelegten Sinusspannung.}
  \label{fig:lcrg}
\end{figure}


\subsection{Messung der Dämpfungskonstante}

Für die Messung der frequenzabhängingen Dämpfung wird eine Rechteckspannung
jeweils über ein kurzes Kabel (ca.\ \SI{25}{\centi\meter}) und das zu untersuchende \SI{85}{\meter}-Kabel in ein Oszilloskop gespeist.

Da eine Rechteckspannung durch eine Fourierreihe aus vielen Frequenzen erzeugt wird,
kann die Dämpfungskonstante in einer Messung für viele Frequenzen bestimmt werden.
Hierzu werden die gemessenen Spannungen im Oszilloskop Fourier-transformiert.

Die eingespeisten Rechteckspannungen sind in \autoref{fig:attenuation_signal} dargestellt.
Es lässt sich erkennen, dass die Form für das kurze Kabel deutlich \enquote{eckiger} ist als bei dem Signal, dass durch das \SI{85}{\meter}-Kabel geleitet wurde.
Dies deutet auf eine Abschwächung der höheren Frequenzen hin.

\begin{figure}
  \centering
  \includegraphics{attenuation_signal.pdf}
  \caption{%
    Rechteckspannung im Oszilloskop nach Leitung durch das kurze bzw.\ \SI{85}{\meter}-Kabel.%
  }\label{fig:attenuation_signal}
\end{figure}

In \autoref{fig:attenuation_fft} sind die Fourier-transformierten Spannungswerte gegen die Frequenzen aufgetragen.
Die Dämpfungskonstante wird für jedes lokale Maxima im Frequenzraum bestimmt,
da das Oszilloskop bereits die Amplitude $A$ in \si{\deci\bel} ermittelt, ergibt sich die Dämpfungskonstante $α$ zu
\begin{equation}
  α / \si{\deci\bel} = (A_{\SI{85}{\meter}} - A_\text{kurz}) / \si{\deci\bel}
\end{equation}

\begin{figure}
  \centering
  \includegraphics{attenuation_fft.pdf}
  \caption{%
    Fourier-transformiertes Signal mit lokalen Maxima für beide Kabel, sowie die für die lokalen Maxima bestimmte Dämpfungskonstante $α$.
  }\label{fig:attenuation_fft}
\end{figure}

\subsection{Bestimmung der Kabellänge}

\begin{figure}
  \centering
  \includegraphics{length_measurement.pdf}
  \caption{%
    Signal des NIM-Pulses mit Reflektion, der Zeitabstand zwischen den beiden Pulsen wird zur Bestimmung der Kabellänge genutzt.
  }\label{fig:length}
\end{figure}

\begin{table}
  \centering
  \caption{%
    Laufzeitunterschied $\increment t$ und
    Kabellänge $l$ für die drei untersuchten Kabel
  }\label{tab:label}
  \input{build/length.tex}
\end{table}

\subsection{Vermessung von unbekannten Abschlusswiderständen}
Es sollte der Spannungsverlauf für verschiedene Abschlüsse gemessen werden. Mithilfe des Oszilloskops wurde der Zeitverlauf der Spannung für
drei unbekannte Abschlusswiderstände angeschlossen. Für Kasten 1 der Anschluss mit der Nummer 4. Für Kasten 2 jeweils Anschlüsse 2 und 6.
Die aufgezeichneten Spannungsverläufe werden mit denen verglichen, welche in Tabelle 2 der Anleitung~\cite{anleitung} dargestellt sind.
Die Spannungskurve für den Anschluss 4 von Kasten 1 ähnelt dem Verlauf von Reihe 4 in Tabelle 2.
Es handelt sich also mutmasslich um einen Abschlusses welcher aus einer Reihenschaltung aus ohmschen Widerstand und Induktivität besteht (siehe
Abbildung~\ref{fig:end_1}).

\begin{figure}
    \centering
    \begin{subfigure}[b]{0.3\textwidth}
        \begin{circuitikz}[american voltages]
\draw
  % stator circuit
  (0,0) to [open, v^>=$U$] (0,3) %
  to [short, *- ] (2,3) %
  % to [R, l=$R_L$] (3,3) %
  to [L, l=$L$] (6,3) %
  to [C, l=$C$] (6,0) %

  (6,3)    to [short, -*] (10,3) %

  (0,0) to [short, *-*] (10,0);
\end{circuitikz}

        \caption{Widerstand in Reihe mit Induktivität}
        \label{fig:end_1}
    \end{subfigure}
    ~ %add desired spacing between images, e. g. ~, \quad, \qquad, \hfill etc.
    \begin{subfigure}[b]{0.3\textwidth}
        \begin{circuitikz}
\draw
  (0,0) to [C] (0,1.5) %
  to [R] (0,3); %
\end{circuitikz}

        \caption{Widerstand in Reihe mit Kapazität}
        \label{fig:end_2}
    \end{subfigure}
    ~ %add desired spacing between images, e. g. ~, \quad, \qquad, \hfill etc.
    \begin{subfigure}[b]{0.3\textwidth}
        \begin{circuitikz}
\draw
  (0,0) to [short] (4,0)%
  (1,0) to [R] (1,3) %
  (3,0) to [C] (3,3) %
  (0,3) to [short] (4,3); %

\end{circuitikz}

        \caption{Widerstand parallel zur Kapazität}
        \label{fig:end_3}
    \end{subfigure}
    \caption{Die mutmasslichen Abschlüsse der Koaxialkabel.}\label{fig:ends}
\end{figure}

\begin{figure}
  \centering
  \includegraphics{unknown.pdf}
  \caption{%
    Signalspannung am Oszilloskop für drei unbekannte Abschlusswiderstände.
    Kasten 1 der Anschluss mit der Nummer 4 in der obersten Spannungskurve. Kasten 2 jeweils Anschlüsse 2 und 6 in den
    unteren Spannungskurven.
  }\label{fig:unknown}
\end{figure}

\subsection{Mehrfachreflektion}

\begin{figure}
  \centering
  \includegraphics{multiple_reflection.pdf}
  \caption{%
    Signale einer Recheckspannung (oben) und eines NIM-Pulsers (unten) für eine Reihenschalten eines \SI{50}{\ohm}-Kabels mit einem \SI{75}{\ohm}-Kabel.
    Besonders beim NIM-Pulser lassen sich Mehrfachreflektionen erkennen.
  }\label{fig:multiple_reflection}
\end{figure}
