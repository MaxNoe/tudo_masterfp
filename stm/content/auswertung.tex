\section{Auswertung}
\label{sec:Auswertung}

\begin{figure}
  \centering
  \includegraphics{gold.pdf}
  \caption{Rastertunnelmikrosp-Aufnahme einer Goldprobe.}
  \label{fig:gold}
\end{figure}


\begin{figure}
  \centering
  \includegraphics{height_profile.pdf}
  \caption{Höhenprofil bei $x=\SI{4.88}{\nano\meter}$, entsprechend einer Spalte aus \autoref{fig:gold}. Vor und nach der Stufenkante wurde eine lineare Regression durchgeführt.}
  \label{fig:profile}
\end{figure}

Um die höhe einer Stufenkante aus dem Bild der Goldprobe zu bestimmen, wird ein Höhenprofil bei $x=\SI{4.88}{\nano\meter}$ betrachtet.
Vor und hinter der Stufenkante bei  $y \approx \SI{260}{\nano\meter}$ wird eine Lineare Regression durchgeführt.
Der Abstand der beiden Geraden berechnet sich zu
\begin{equation}
  \increment h = \cos(\arctan(m)) \cdot (b_1 - b_2).
\end{equation}
Hierbei bezeichnet $m$ die Steigung der beiden Geraden, sowie $b_1$ und $b_2$ den $y$\=/Achsenabschnitt bezeichnen.

Aus dem Fit ergeben sich:
\input{fitresults.tex}
und damit berechnet sich die Höhe der Stufenkante zu
\begin{equation}
  \increment h = \input{height.tex}.
\end{equation}
