\section{Auswertung}
\label{sec:Auswertung}

Die von der Software des Rastertunnelmikroskops geschriebenen Dateien enthalten einen Kopf mit Metadaten, der das Format von \texttt{.ini} aufweist.
Nach einem Steuerzeichen (\texttt{\#!}) folgen die Daten in einem binären Block als 16-bit \texttt{float}s.
Zum Einlesen der Daten wird daher eine Kombination aus dem \texttt{configparser} Modul der \texttt{python} Standardbibliothek~\cite{configparser} und \texttt{numpy}~\cite{numpy} verwendet.~\cite{readnid}

\subsection{HOPG}
\begin{figure}
  \centering
  \includegraphics{hopg_uncorrected}
  \caption{%
    Tunnelstrom $I$ (oben) und Höhe $z$ (unten) für die beiden Aufnahmen von HOPG.
    Es lässt sich deutlich eine schiefe Ebene in der Höhe erkennen.%
  }\label{fig:uncorrected}
\end{figure}

Zur Berechnung der Gitterkonstanten und Winkel in Graphit werden zwei Bilder des Rastertunnelmikroskops genutzt, die mit einer unterschiedlich Messrichtung aufgenommen wurden.
Höhenverteilung und Tunnelstrom für diese beiden Aufnahmen ist in \autoref{fig:uncorrected} aufgetragen.

In der gemessenen Höhenverteilung zeigt sich ein deutliches Gefälle, um dies zu korrigieren, wird eine Ebene an die Höhe angepasst und das Ergebnis abgezogen.
In \autoref{fig:hopg_up} und \autoref{fig:hopg_down} sind die so korrgierten Messwerte dargestellt.

Zum auffinden der lokalen Maxima wird die Bibiliothek \texttt{scikit-image}~\cite{skimage} genutzt.
Per Hand werden eine Reihe und eine Diagonale zur Ermittlung des Winkels und der Gitterkonstanten gewählt.
Auf den ausgewählten Punkten wird eine lineare Regression durchgeführt.
Aus den Steigungen dieser beiden Geraden ergibt sich der Winkel zwischen den Basisvektoren durch
\begin{equation}
  α = \arctan(m_1) -\arctan(m_2).
\end{equation}
Für die Aufnahme in \enquote{up}-Richtung ergibt sich:
\begin{equation}
  \input{./build/grid_angle_up.tex}.
\end{equation}
Für die Aufnahme in \enquote{down}-Richtung ergibt sich:
\begin{equation}
  \input{./build/grid_angle_down.tex}.
\end{equation}
Für die Gitterkonstanten in die jeweilige Richtung ergibt sich für die \enquote{up}-Aufnahme:
\begin{align}
  g_\text{horizontal} &= \input{./build/grid_constant_horizontal_up.tex} \\
  g_\text{diagonal} &= \input{./build/grid_constant_diagonal_up.tex} \\
\shortintertext{und}
  g_\text{horizontal} &= \input{./build/grid_constant_horizontal_down.tex} \\
  g_\text{diagonal} &= \input{./build/grid_constant_diagonal_down.tex}
\end{align}
für die \enquote{down}-Aufnahme.

\begin{figure}
  \centering
  \includegraphics{hopg_up.pdf}
  \caption{%
    Rastertunnel-Aufnahme der HOPG Oberfläche in \enquote{up}-Richtung.
    In weiß sind die gefundenen lokalen Maxima dargestellt. Die farbig markierten Punkte wurden zur Berechnung der Gittervektoren verwendet.
    Die Linien zeigen das Ergebnis der Linearen Regression.
  }\label{fig:hopg_up}
\end{figure}

\begin{figure}
  \centering
  \includegraphics{hopg_down.pdf}
  \caption{%
    Rastertunnel-Aufnahme der HOPG Oberfläche in \enquote{down}-Richtung.
  }\label{fig:hopg_down}
\end{figure}

\subsection{Gold}

\begin{figure}
  \centering
  \includegraphics{gold.pdf}
  \caption{Rastertunnelmikrosp-Aufnahme einer Goldprobe.}
  \label{fig:gold}
\end{figure}

Um die höhe einer Stufenkante aus dem Bild der Goldprobe zu bestimmen, wird ein Höhenprofil bei $x=\SI{4.88}{\nano\meter}$ betrachtet.
Vor und hinter der Stufenkante bei  $y \approx \SI{260}{\nano\meter}$ wird eine Lineare Regression durchgeführt.
Der Abstand der beiden Geraden berechnet sich zu
\begin{equation}
  \increment h = \cos(\arctan(m)) \cdot (b_1 - b_2).
\end{equation}
Hierbei bezeichnet $m$ die Steigung der beiden Geraden, sowie $b_1$ und $b_2$ den $y$\=/Achsenabschnitt bezeichnen.

\begin{figure}
  \centering
  \includegraphics{height_profile.pdf}
  \caption{%
    Höhenprofil bei $x=\SI{4.88}{\nano\meter}$,
    entsprechend einer Spalte aus \autoref{fig:gold}.
    Vor und nach der Stufenkante wurde eine lineare Regression durchgeführt.}\label{fig:profile}
\end{figure}


Aus dem Fit ergeben sich:
\input{fitresults.tex}
und damit berechnet sich die Höhe der Stufenkante zu
\begin{equation}
  \increment h = \input{height.tex}.
\end{equation}

