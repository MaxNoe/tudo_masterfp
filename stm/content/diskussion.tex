\section{Diskussion}\label{sec:Diskussion}

Die ermittelte Höhe der Stufenkante in der Goldprobe von
\begin{equation*}
  \increment h = \input{height.tex}.
\end{equation*}
 Das ist etwa \input{step_factor.tex} mal der Netzebenenabstand von Gold von
 \begin{equation*}
   s = \frac{\SI{407.82}{\pico\meter}}{\sqrt{3}} = \SI{235.45}{\pico\meter} \cite{}.   
 \end{equation*}


Es kann sich bei der Stufenkante also um genau eine fehlende Atomschicht handeln. Ein möglicher Fehler bei der Messung der Höhenprofile
kommt eventuell von der Hysterese der Piezokristalle. Durch die Hyterese müssen beim Heben bzw\. Senken der Spitze unterschiedliche Spannungen angelegt werden
um die gleiche Höhe der Spitze und damit den gleichen Tunnelstrom zu erhalten.


Die Messung der Gitterkonstante an der HOPG Oberfläche ist veträglich mit dem Literaturwert von \SI{2.46}{\angstrom}. Zumindest bei Messung in horizontaler Richtung.
Der Korrekturfaktor für die Messung in $y$-Richtung beträgt \input{build/correction_factor_up.tex} in \enquote{up} und \input{build/correction_factor_down.tex} in
\enquote{down} Richtung. Anscheinend sind die Piezokristalle nicht optimal kalibriert.
Aufgrund der scheinbar korrekten Messung in $x$-Richtung wurde der Korrekturfaktor für diese Richtung als \num{1} angenommen. Auch ist die Steigung der angepassten Geraden verträglich mit \num{0}.
